\documentclass[notitlepage, twocolumn]{revtex4}
\begin{document}
\title{Planes}
\author{Rose Gibson}
\email{rgibson2@wellesley.edu}
\affiliation{KPNO REU 2015 and Wellesley College, Wellesley College, Wellesley, MA 02481}
\author{Chuck Claver (Advisor)}
\affiliation{National Optical Astronomy Observatory, Tucson, AZ 85719}

\begin{abstract}
abstract goes here eventually
\end{abstract}
\maketitle        

\section{Introduction}

The Large Synoptic Survey Telescope (LSST) is scheduled to be operational by 2022, and will sweep the southern sky approximately once a week. One of the goals is to observe changes in brightness over a short period of time for objects as far as supernovae or as close as near Earth asteroid, so the cadence of the telescope becomes extremely important. The survey will be able to use local weather and seeing data to determine which regions of the sky to observe, and despite being located in the remote Cerro Pach$\acute{o}$n mountain, there is the additional problem of planes flying overhead and the contrails streaming behind them.

Condensation Trails, or contrails, are artificial clouds that form behind aircraft. Their formation is triggered by water vapor from the engine exhaust and change in air pressure due to the air vortexes. Most contrails disappear within a few minutes [Jenson], however air-turbulence and differences in wind speed along the flight rack can spread the contrails into fully formed cirrus clouds that can last for up to an hour [faa.gov] The goal of this summer's research was to track the airplanes and feed the information into the LSST cadence system.

Every commercial aircraft emits a transponder signal at 1090MHz called the Automatic Dependent Surveillance-Broadcast System (ADS-B) which can be received by any antenna set to that frequency. With a Digital Video Broadcasting-Terrestrial (DVB-T) device and a software called dump1090, the radio signal can be converted into useable data. The positions of the planes are plotted on images from LSST's All Sky Camera. 
\section{Methods}
\begin{itemize}
\item Figure of Tucson setup
\item Figure of Chile setup
\item Describe the Tucson setup
\item Describe the Chile setup
\end{itemize}

\section{Results}
\begin{itemize}
\item Statistical results of airplane frequency
\item Plots that show range of antenna (Tucson vs Cerro Pach$\acute{o}$n)
\item Plane data plotted on All Sky image with contrails
\end{itemize}
\section{Discussion}
\begin{itemize}
\item Discuss what the stats mean for LSST
\end{itemize}
\section{Conclusion}

\end{document}
